\begin{abstractzh}
{我們提出了一種新的特殊相對論流體力學(SRHD)程式,能夠同時處理共存超相對論熱和非相對論冷的氣體。
我們通過在 SRHD 求解器中設計一種可以轉換原始和守恆變量的新算法來實現這一點,
該算法結合了一個涵蓋相對論和非相對論機制的真實理想氣體狀態方程。
該程式可以處理涉及洛倫茲因子高達 $10^6$ 的問題,並最佳地避免數值災難性相消。
此外,我們已將新的 SRHD 求解器合併至 GAMER (https://github.com/gamer-project/gamer) 中,
以支持自適應網格細化和混合 OpenMP/MPI/GPU 平行化。
它在單個 Tesla P100 GPU 上實現了每秒 $7\times10^7$ 更新網格的峰值性能,
並且可以很好地擴展到 2048 個 GPU。我們將此程式應用於兩個有趣的天體物理學問題:
(1)來自不對稱爆炸源的相對論衝擊波(2)相對論射流中的自我加速和臨邊增亮現象。

費米伽馬射線太空望遠鏡揭示了銀河系中的兩個大氣泡,它們在銀河系中心 (GC)
上方和下方幾乎對稱地延伸約 50 度。最近發現更巨大的義羅西塔氣泡也顯示對於 GC 的這種對稱性,
表明它們可能源自同個發生於 GC 附近的事件。先前對氣泡形成的模擬假設
活躍星系核 (AGN) 噴流是垂直於銀河盤面;然而,一般來說,噴流方向和盤面的旋轉軸之間沒有特別的相關性。
我們使用包括宇宙射線 (CR) 和熱氣體在內的三維特殊相對論流體動力學模擬,
並表明事實上, 氣泡對稱性是因為圓盤內的星際物質緻密塊狀氣體破壞了射流的准直性(以下稱為“失敗的射流”),
導致失敗的噴流形成熱氣泡。隨後分層大氣中的浮力使熱氣泡
垂直拉長,形成現今觀測到的對稱費米氣泡和義羅西塔氣泡,(統稱:星系氣泡)。
具體來說,我們發現
(1)儘管在一千兩百萬年前,從 GC 發出的相對論射流相對於銀河系的旋轉軸小於等於 45 度
的各種角度,但銀河氣泡仍然與軸對齊;模擬的銀河氣泡幾乎是對於 GC 對稱的,儘管雙極噴流與銀河旋轉軸成 45度角;
(2) 義羅西塔氣泡的邊緣對應於由熱氣泡驅動的前向衝擊波;
(3) 緊隨其後的前向衝擊波是在費米氣泡邊緣附近的糾纏接觸間斷(contact discontinuity)
與外部介質壓力平衡的湍流和高溫($\sim$2 keV)等離子體組成的氣泡;
(4) 假設使用輕子模型,我們發現觀測到的伽馬射線氣泡和微波霧霾可以用 CR 冪律指數 2.4 與現今觀測到的數據擬合;
(5) 從現今到短壽命噴流發射起初之間的一千兩百萬年, 這個時間跨度似乎是觀測到的 GC
中央分子區形成的時間尺度。模擬和觀察到的多波長特徵之間的廣泛一致性表明,
由傾斜的 AGN 失敗噴流形成銀河氣泡是一種合理的情況。}
\bigbreak
\noindent \textbf{關鍵字:}{\, \makeatletter 相對論性流體、數值方法、費米氣泡、義羅西塔氣泡 \makeatother}
\end{abstractzh}

\begin{abstracten}
We present a new special relativistic hydrodynamics (SRHD) code capable of handling coexisting
ultra-relativistically hot and non-relativistically cold gases.
We achieve this by designing a new algorithm for conversion between primitive and
conserved variables in the SRHD solver, which incorporates a realistic ideal-gas
equation of state covering both the relativistic and non-relativistic regimes.
The code can handle problems involving a Lorentz factor as high as $10^6$ and optimally
avoid the catastrophic cancellation. In addition, we have integrated this new SRHD solver
into the code \textsc{gamer} (\url{https://github.com/gamer-project/gamer}) to support
adaptive mesh refinement and hybrid OpenMP/MPI/GPU parallelization.
It achieves a peak performance of $7\times 10^{7}$ cell updates per second on a
single Tesla P100 GPU and scales well to 2048 GPUs. We apply this code
to two interesting astrophysical applications: (1) an asymmetric explosion source
on the relativistic blast wave and (2) the flow acceleration and limb-brightening of
relativistic jets.

The \textit{Fermi Gamma-Ray Space Telescope} reveals two large bubbles in the Galaxy,\
which extend nearly symmetrically $\sim50^{\circ}$ above and below the Galactic center (GC).\
The recent discovery of giant eROSITA bubbles also shows such a symmetry about the GC,
suggesting that they may originate from a single GC activity.\
Previous simulations of bubble formation that invoke active galactic nucleus (AGN) jets\
have assumed that the jets are vertical to the Galactic plane;\
however, in general there does not need to be a correlation between
the jet orientation and the rotational axis of the Galactic disk.\
Using three-dimensional (3D) special relativistic hydrodynamic (SRHD) simulations that\
include cosmic rays (CRs) and thermal gas, we show that the dense clumpy gas within disk interstellar medium\
disrupts jets collimation and confinement (``failed jets" hereafter),
which in turn causes the failed jets to form
hot bubbles. Subsequent bouyancy in the stratified atmosphere renders them vertical to form
the symmetric \textit{Fermi} and eROSITA bubbles (collectively, Galactic bubbles).
Specifically, we find that\
(1) despite the relativistic jets emanated from the GC 12 Myr ago are at various angles
$\leq 45^{\circ}$ with respect to
the rotational axis of the Galaxy, the Galactic bubbles nonetheless appear aligned with the axis;
the simulated Galactic bubbles are nearly symmetric about the GC albeit\
the bipolar jets are at an angle $45^{\circ}$ with respect to\
the rotational axis of the Galaxy;\
(2) the edge of the eROSITA bubbles corresponds to a forward shock, driven by the hot bubble;
(3) followed by the forward shock is a tangling contact discontinuity at the edge of the \textit{Fermi} bubbles
composed of turbulent and high-temperature ($\sim2$ keV)\
plasma in pressure balance with the external medium;
(4) assuming a leptonic model we find that the observed gamma-ray bubbles and microwave haze can be reproduced with
a best-fit CR power-law index of 2.4;
(5) the 12 Myr time span between the present and the launch of short-lived jets appears to be the appropriate time
scale for the observed GC Central Molecular Zone to form.
The broad agreement between the simulated and the
observed multi-wavelength features suggests that forming the Galactic bubbles by
oblique AGN failed jets is a plausible scenario.

\bigbreak
\noindent \textbf{Keywords:}{\, \makeatletter relativistic hydrodynamics, numerical method, \textit{Fermi}/eROSITA bubbles\makeatother}
\end{abstracten}

\begin{comment}
\category{I2.10}{Computing Methodologies}{Artificial Intelligence --
Vision and Scene Understanding} \category{H5.3}{Information
Systems}{Information Interfaces and Presentation (HCI) -- Web-based
Interaction.}

\terms{Design, Human factors, Performance.}

\keywords{Region of interest, Visual attention model, Web-based
games, Benchmarks.}
\end{comment}
