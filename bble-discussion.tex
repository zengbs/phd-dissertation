\chapter{Discussion}
\label{c:discussion}

%\section{Discussion}
As a past violent event at the GC must destroy the Central Molecular Zone (CMZ)
but the CMZ is currently either in a quasi-steady state or
presently increasing \citep{Crocker2012,Krumholz2015,Sormani2019};
we conclude that the time since the violent event must be ``slightly" shorter
than the reconstruction time of CMZ on a timescale of
$M_{\text{CMZ}}/\dot{M}_{\text{inflow}}$, where $M_{\text{CMZ}}$ is
the total mass of molecular gas within $R\sim300$ pc,
and $\dot{M}_{\text{inflow}}$ is the gas inflow rate from the Galactic disc at
$R\sim3$ kpc down to the outskirts of the CMZ.
Since $M_{\text{CMZ}}$ and $\dot{M}_{\text{inflow}}$ have been reported to be in the range
$2-6\times10^{7}$ M$_{\odot}$ \citep{Dahmen1998,Ferrire2007} and
$0.4-2.7$ M$_{\odot}$ yr$^{-1}$ \citep{Crocker2012,Sormani2019}, respectively;
we estimate that the CMZ formation is about a couple of 10 Myrs, similar to
12 Myr required for our simulated bubbles to reach the desired morphology.

On the other hand, our modeled gamma-ray and microwave spectra assumed that
the underlying CRe spectrum is spatially uniform with hard spectral index 2.4;
however, the high energy CRe severely suffer from synchrotron and IC losses,
during their passage through the magnetic and radiation fields, respectively, within the Galaxy.

The typical synchrotron and IC cooling time scale of high energy ($\sim100$ GeV) CRe
in Milky Way is $\sim$ 1 Myr \citep{Yang2017}, ten times shorter than
12 Myr suggested by our simulation.
Therefore, the CRe generating the gamma-ray emission would need to be re-accelerated
by in-situ acceleration mechanisms such as shocks or turbulence.
Since the forward shock is rather far away from the gamma-ray bubbles,
re-acceleration is more likely associated with turbulence \citep{Mertsch2011,Mertsch2019},
possibly in balance with the IC and synchrotron cooling. We will investigate the
competition between stochastic acceleration and radiative cooling in future work.

Overall, our results imply that in the oblique-jet scenario,
whatever the true re-acceleration and cooling mechanisms are,
the CRe spectrum at the present time has to be spatially uniform with
a spectral index of 2.4 in order to fit the observed spectra.



%\begin{equation}
%   \frac{dE}{dt}
%     = -\frac{4}{3}\sigma_{T}c\beta_{\text{e}}^2\gamma_{\text{e}}^2
%     \left(U_{\text{photon}}+U_{B}\right),
%\end{equation}
%where $\beta_{\text{e}}=v_{\text{e}}/c$, $U_{\text{photon}}$ and $U_{B}$ are
%energy density of ISRF and magnetic field, respectively.



