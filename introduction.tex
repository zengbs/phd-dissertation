\chapter{Introduction}
\label{c:intro}

Many high energy astrophysical problems involve relativistic flows.
The problems include, for example, collimated jets in active galactic nuclei (AGN)
(\citealt{Chiueh-1991ApJ...377..462C};
\citealt{Chiueh-1992ApJ...394..459L};
\citealt{Blandford2018}),
collapsar models of long-duration gamma-ray bursts \citep{LongGRB},
magnetized relativistic winds and nebulae from pulsars
(\citealt{1984ApJ...283..694K}; \citealt{1984ApJ...283..710K};
\citealt{Chiue-PhysRevLett.63.113}; \citealt{Chiueh_1998}), and mildly relativistic
wide-angle outflows in neutron star mergers
(\citealt{NM2}; \citealt{NM1}; \citealt{NM3}; \citealt{NM4}).
The full scope of these problems generally involves substantial temperature changes between
jets (winds) and ambient gases. For this reason, the pioneering works of
\cite{Taub}, \cite{TM_EOS}, and \cite{Compare_TM_EOS} suggested Taub-Mathews equation of
state (TM EoS) that approximates the exact EoS \citep{Synge} for
ultra-relativistically hot (high-$T$ hereafter) gases coexisting with non-relativistically
cold (low-$T$ hereafter) gases.

In addition, \cite{Noble_2006} first compared the accuracy of several schemes for
recovering primitive variables in the Riemann problems by means of self-checking tests
(see Appendix \ref{Appendix:Numerical error analysis} for details).
\cite{NR_Limit} further proposed an inversion scheme for an arbitrary EoS and
suggested that directly evolving the reduced energy density (i.e. the energy density
subtracting the rest mass energy density from the total energy density) can avoid
catastrophic cancellation in the non-relativistic limit.
However, very few studies have systematically investigated how serious the catastrophic
cancellation bears upon simulation results. This is partially due to the lack of
exact solutions with which numerical results can be compared.

In this paper, we propose a new numerical scheme for conversion between primitive and
conserved variables in the presence of both high-$T$ and low-$T$ gases.
The new scheme is carefully tailored to avoid catastrophic cancellation.
To verify its accuracy, we numerically derive the exact solutions of two relativistic
Riemann problems with the TM EoS and compare with the simulation results.
It demonstrates that our new special relativistic hydrodynamics (SRHD) code can
minimize numerical errors compared with conventional methods.

We have integrated this new SRHD solver into the code \textsc{gamer}
(\citealt{gamer-1}; \citealt{gamer-2}) to facilitate GPU acceleration and adaptive mesh
refinement (AMR). This new code, \textsc{gamer-sr}, yields good weak and strong
scalings using up to 2048 GPUs on \texttt{Piz-Daint}, the supercomputer at the Swiss
National Supercomputing Centre (CSCS). Finally, we present two astrophysical applications,
an asymmetric explosion and self-accelerating jets, to demonstrate the capability of
this new code in extreme conditions. All simulation data are analysed and visualized
using the package \texttt{yt} \citep{YT}.

This paper is organized as follows. We introduce the equation of state and our new scheme
for conversion between primitive and conserved variables in Chapter
\ref{c:special-relativistic-hydrodynamics}. In Chapter \ref{c:numerical-methods}, we describe
numerical methods, including the AMR structure, GPU acceleration, flexible time-steps,
and correction of unphysical results. In Chapter \ref{c:test-problems} and
\ref{c:performance-scaling}, we conduct numerical experiments to demonstrate the accuracy in
both the non-relativistic (NR) and ultra-relativistic (UR) limits, the performance scalability,
as well as the limitation of \textsc{gamer-sr}. Finally, we present two astrophysical
applications in Chapter \ref{c:astrophysical-applications} and draw the
conclusion in Chapter \ref{c:conclusions-1}.

Note that the speed of light and the Boltzmann constant are hard-coded to
$1$ in \textsc{gamer-sr}. However, these physical constants are retained in this paper,
except in Appendices, for dimensional consistency.
