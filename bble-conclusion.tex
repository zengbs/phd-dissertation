
\chapter{Conclusions for part 2}
\label{c:conclusions-2}

%\section{Conclusions}
\label{Conclusions}
In this work, we introduce a thin, dense disk composed of clumpy ISM\
to stall and thermalize the oblique jets for an outburst event from
the central SMBH in the Milky Way Galaxy 12 Myrs ago.
We investigate the properties of the Galactic bubbles and the microwave\
haze using 3D SRHD simulations of CR jet injections from\
the SMBH assuming the leptonic model. The important findings are summarized as follows.

\begin{itemize}


%\item The clumpiness of the ISM disk has an insignificant impact on the\
%      overall dynamic of the Galactic bubbles.\
%      The development of the reverse shocks (the innermost bubbles) is always
%      associated with the dense disk,\
%      without the disc, only asymmetric forward shocks exist.
%      {\color{red} KY: I removed the second point about the innermost bubbles for now, because I think the following two points should be separated: (1) inclusion of the disk is critical for forming symmetric Galactic bubbles, (2) we found a pair of innermost bubbles which are reverse shocks (and perhaps comment on their properties and/or observable signatures). PH: done! }
\item The development of the expanding forward-reverse shock pair is always associated with the dense disk.
%the dense disk, although the clumpiness of the disk has an insignificant impact on the\
%overall dynamic of the Galactic bubbles.
%Furthermore,
In the absence of the disk, the reverse shock is absent,
indicating the inclusion of the disk is critical for forming the innermost bubbles.

\item The forward-reverse shock pair heats the turbulent plasma considerably ($\sim2$ keV).
%as the plasma is bracketed between their downstream.
There exists a
%dense shell that fits closely to the reverse shock because the gas is
%immediately compressed as it passes outward through the reverse shock while coinciding with the low gas density further downstream.
dense shell immediately downstream of the reverse shock, a situation reminiscent of a supernova shell.

\item eROSITA bubbles coincide with the forward shock front
      originally driven by short-lived bipolar jets
      for a duration of 1.2 Myr, where the bubbles later significantly
      expanded into the stratified atmosphere to reach the present 12kpc height.
      The overall extent of the simulated X-ray bubbles is comparable to
      that of the eROSITA bubbles, though not as limb-brightened.
      Future models including shock-accelerated CRs may help to resolve this issue by
      increasing the compressibility of the fluid and
      enhancing the thermal Bremsstrahlung emissivity
      at the edge of the X-ray bubbles.
      %The simulated eROSITA bubbles are not as limb-brightened as the observation.\
      %A possibility to enhance the X-ray emission is to include shock-accelerated CRs near the shock,\
      %in which CRs could increase the compressibility of the fluid,\
      %resulting in the enhanced thermal Bremsstrahlung emissivity that is proportional to density squared.
%\item Followed by the forward shock is a turbulent, hot plasma ($\sim2$ keV) filled
      %with the injected CRs in pressure balance with the shocked medium,
      %which we interpret as the \textit{Fermi} bubbles.\
      %The similarity of the morphology, temperature, and gamma-ray spectrum between\
      %simulated and observed bubbles suggests that\
      %the edge of the \textit{Fermi} bubbles are likely associated with the contact discontinuity between
      %the turbulent, hot plasma and the post-shock medium.
\item Downstream of the reverse shock is filled with hot ($\sim2$ keV) and highly turbulent
      plasma brought from the disk. The interface between the downstream materials of reverse and
      forward shocks lies a contact discontinuity, which defines the edge of the Fermi bubble.
      The surface of the simulated bubbles is
      not as smooth as the observed ones; inclusion of magnetic fields in the future may help suppress
      the instabilities at the bubble surface due to the mangetic draping effect.
\item Assuming a power-law CRe energy spectrum ranging from 0.5 MeV to 560 GeV,
      where the spectrum is space-independent,
      %Assuming a leptonic model with a spatially uniform, power-law CRe spectrum
      %ranging from 0.5 MeV to 562.1 GeV,
      we showed that the observed gamma-ray and microwave spectra
      can simultaneously reproduced. The best-fit CRe power-law index is found to be 2.4.
\item The elapsed time of 12 Myr is required for the simulated bubbles
      to reach the desired morphology.
      This time scale is appropriate for the observed Central Molecular Zone, 200 pc
      in radius surrounding the GC, to form. However, it is 10 times longer than
      %The elapsed time of 12 Myr required for the simulated bubbles
      %to reach the desired morphology is ten times longer than
      the typical synchrotron and IC cooling time scale of
      high energy ($\sim100$ GeV) CRe. Thus, re-acceleration of CRe by shocks or turbulence must be considered
      in this model. Since the forward shock is rather far away from the gamma-ray bubbles,
      stochastic acceleration of CRs by turbulence appears to be more plausible.\
      We will investigate the competition between stochastic acceleration and radiative cooling
      in a future work.
      %{\color{red} KY: This item is to be modified according to the newly added text in the Discussion section. PH: done!}
\item The Galactic bubbles are observed nearly symmetric about the Galactic plane albeit\
      the bipolar jets are oblique with respect to the disk normal.\
      We showed that inclusion of the dense ISM disk (regardless of its clumpiness)
      is an essential ingredient for producing the symmetric Galactic bubbles when the jets are oblique.\
      The broad agreement between the simulated and observed multi-wavelength features\
      demonstrates that oblique failed jets are a plausible
      %scenario of forming the Galactic bubbles,
      scenario for the formation of the Galactic bubbles,
      which relieve the caveat of earlier jet models where jets need to be vertical.
      %that the jets must be vertical to the Galactic plane.\

\end{itemize}


