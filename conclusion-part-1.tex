\chapter{Conclusions for part 1}
\label{c:conclusions}
 In this paper, we have presented a novel special relativistic hydrodynamics code, \textsc{gamer-sr}, which incorporates a new, well-tailored conversion scheme (cf. \Cref{fig:flowchart}) between primitive and conserved variables, together with the Taub-Mathews equation of state (TM EoS) covering both the ultra-relativistic and non-relativistic limits. The new scheme adopts the four-velocity $U^{j}$, the reduced energy density $\tilde{E}$, and the reduced enthalpy $\tilde{h}$ to effectively avoid the catastrophic cancellation in subsonic flows at all finite temperature, including the particularly challenging low-temperature regime, with errors decreasing as $\mathscr{M}^2\epsilon_{\text{machine}}$ when $\mathscr{M}\gg1$.

 We have numerically derived the exact solution of a Riemann problem covering both extreme cold and ultra-relativistically hot gases with the TM EoS. Simulation results using our new scheme are in very good agreement with the exact solution in both the ultra-relativistic and non-relativistic regimes. (cf. \Cref{fig:non-relativistic shock tube}). In comparison, the catastrophic cancellation arising from the original (unoptimized) scheme can be much more severe than the truncation error in the non-relativistic limit, especially in the region swept by a traveling contact discontinuity.

The new scheme has been integrated into the code \textsc{gamer} to facilitate the hybrid OpenMP/MPI/GPU parallelization and adaptive mesh refinement. Thanks to that, the performance of the root-finding iterations in the TM EoS can be significantly improved by GPU. The parallel efficiency using 2048 computing nodes is measured to be $\sim$ 45 per cent for strong scaling (cf. \Cref{fig:strong scaling}) and $\sim$ 75 per cent for weak scaling (cf. \Cref{fig:weak scaling}) on the \texttt{Piz-Daint} supercomputer.

\textsc{gamer-sr} has been demonstrated to be able to handle ultra-relativistic flow with a Lorentz factor as high as $10^6$. However, we also find that the Cartesian grids can lead to artificial dissipation when the direction of a high Mach number flow is not aligned with grids. This problem cannot be mitigated by increasing spatial and temporal resolution.

 We have examined two astrophysical problems with coexisting relativistically hot and cold gases to demonstrate the power of \textsc{gamer-sr}. The first problem deals with a relativistic blast wave with a triaxial source. Not only do we find that the code is able to capture the ultra-relativistic strong shock very well, but we also discover a simple rule governing how the triaxiality of the blast wave diminishes as a function of the blast wave radius.

 The second problem addresses the flow acceleration and limb-brightening of a relativistic AGN jet. We find that the jet, from its head to its source, is always enclosed inside a turbulent cocoon. The jet is accelerated all the way up to the first confinement point, where an internal shock appears. We attribute such flow acceleration to the relativistic Bernoulli's law. In addition, the synchrotron limb-brightening is found to be caused by the jet transverse boundary shock, outside which the post-shock cosmic-ray particles are mixed with the turbulent cocoon and give out extra synchrotron emission.

\textsc{gamer-sr} does not implement explicit grid derefinement criteria, dual-energy formalism, and magnetic fields (e.g., \citealt{Stone2020}). The magnetic fields play a critical role in many high energy astrophysical problems, such as the evolution of accretion discs (e.g., \citealt{Blandford1982}), and the interaction between cosmic rays and thermal gases in the Fermi bubbles (e.g., \citealt{Yang2017}). We leave them to future work.


