\begin{abstractzh}
本論文提出了一影像中使用者感興趣區域 (region of interest)
偵測之資料集 (benchmark)。
使用者感興趣區域偵測在許多應用中極為有用,
過去雖然有許多使用者感興趣區域之自動偵測演算法被提出,
然而由於缺乏公開資料集,
這些方法往往只測試了各自的小量資料而難以互相比較。
從其它領域可以發現,
基於公開資料集的可重製實驗與該領域突飛猛進密切相關,
因此本論文填補了此領域之不足,
我們提出名為「Photoshoot」的遊戲來蒐集人們對於感興趣區域的標記,
並以這些標記來建立資料集。
透過這個遊戲,我們已蒐集大量使用者對於感興趣區域的標記,
並結合這些資料成為使用者感興趣區域模型。
我們利用這些模型來量化評估五個使用者感興趣區域偵測演算法,
此資料集也可更進一步作為基於學習理論演算法的測試資料,
因此使基於學習理論的偵測演算法成為可能。

\bigbreak
\noindent \textbf{關鍵字:}{\, \makeatletter \@keywordszh \makeatother}
\end{abstractzh}

\begin{abstracten}
We present a new special relativistic hydrodynamics (SRHD) code capable of handling coexisting
ultra-relativistically hot and non-relativistically cold gases.
We achieve this by designing a new algorithm for conversion between primitive and
conserved variables in the SRHD solver, which incorporates a realistic ideal-gas
equation of state covering both the relativistic and non-relativistic regimes.
The code can handle problems involving a Lorentz factor as high as $10^6$ and optimally
avoid the catastrophic cancellation. In addition, we have integrated this new SRHD solver
into the code \textsc{gamer} (\url{https://github.com/gamer-project/gamer}) to support
adaptive mesh refinement and hybrid OpenMP/MPI/GPU parallelization.
It achieves a peak performance of $7\times 10^{7}$ cell updates per second on a
single Tesla P100 GPU and scales well to 2048 GPUs. We apply this code
to two interesting astrophysical applications: (a) an asymmetric explosion source
on the relativistic blast wave and (b) the flow acceleration and limb-brightening of
relativistic jets.

The \textit{Fermi Gamma-Ray Space Telescope} reveals two large bubbles in the Galaxy,\
which extend nearly symmetrically $\sim50^{\circ}$ above and below the Galactic center (GC).\
The recent discovery of giant eROSITA bubbles also shows such a symmetry about the GC,
suggesting that they may originate from a single GC activity.\
Previous simulations of bubble formation that invoke active galactic nucleus (AGN) jets\
have assumed that the jets are vertical to the Galactic plane;\
however, in general there does not need to be a correlation between
the jet orientation and the rotational axis of the Galactic disk.\
Using three-dimensional (3D) special relativistic hydrodynamic (SRHD) simulations that\
include cosmic rays (CRs) and thermal gas, we show that the dense clumpy gas within disk ISM\
disrupts jets collimation and confinement (``failed jets" hereafter),
which in turn causes the failed jets to form
hot bubbles. Subsequent bouyancy in the stratified atmosphere renders them vertical to form
the symmetric \textit{Fermi} and eROSITA bubbles (collectively, Galactic bubbles).
Specifically, we find that\
(1) despite the relativistic jets emanated from the GC 12 Myr ago are at various angles
$\leq 45^{\circ}$ with respect to
the rotational axis of the Galaxy, the Galactic bubbles nonetheless appear aligned with the axis;
the simulated Galactic bubbles are nearly symmetric about the GC albeit\
the bipolar jets are at an angle $45^{\circ}$ with respect to\
the rotational axis of the Galaxy;\
(2) the edge of the eROSITA bubbles corresponds to a forward shock, driven by the hot bubble;
(3) followed by the forward shock is a tangling contact discontinuity at the edge of the \textit{Fermi} bubbles
composed of turbulent and high-temperature ($\sim2$ keV)\
plasma in pressure balance with the external medium;
(4) assuming a leptonic model we find that the observed gamma-ray bubbles and microwave haze can be reproduced with
a best-fit CR power-law index of 2.4;
(5) the 12 Myr time span between the present and the launch of short-lived jets appears to be the appropriate time
scale for the observed GC Central Molecular Zone to form.
The broad agreement between the simulated and the
observed multi-wavelength features suggests that forming the Galactic bubbles by
oblique AGN failed jets is a plausible scenario.

\bigbreak
\noindent \textbf{Keywords:}{\, \makeatletter relativistic jets, numerical method \makeatother}
\end{abstracten}

\begin{comment}
\category{I2.10}{Computing Methodologies}{Artificial Intelligence --
Vision and Scene Understanding} \category{H5.3}{Information
Systems}{Information Interfaces and Presentation (HCI) -- Web-based
Interaction.}

\terms{Design, Human factors, Performance.}

\keywords{Region of interest, Visual attention model, Web-based
games, Benchmarks.}
\end{comment}
