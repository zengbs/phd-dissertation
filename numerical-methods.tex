\chapter{Numerical methods}
\label{c:numerical-methods}

\section{A GAMER Primer}
Due to the flexibility and extensibility of \textsc{gamer} (\citealt{gamer-1}; \citealt{gamer-2}), the SRHD module directly inherits the AMR structure and the MPI/OpenMP/GPU parallelization framework of hydrodynamics, and therefore we only provide a summary here. We define the base grid resolution as level-0 and the $\ell$th refinement as level-$\ell$, where level-$\ell$ has a spatial resolution $2^{\ell}$ times higher than that of the base level. Data in \textsc{gamer} are always decomposed into patches, each of which consists of $8^3$ cells, and the AMR implementation is realized by constructing a hierarchy of patches in an octree structure. According to user-defined refinement criteria, we can create or remove fine patches under the proper-nesting constraint.

In addition to the refinement criteria provided by the hydrodynamics module, we also implement two refinement criteria for SRHD: the gradient of the Lorentz factor and the magnitude of $|\mathbf{M}|/D$. The former aims to capture the thin and high-$\gamma$ shell in the Sedov-Taylor blast wave, while the latter ensures that the spine region in an over-pressured jet (cf. \Cref{fig:Limb_brightened_jet}) can be fully resolved. For all refinement criteria, the refinement thresholds on different levels can be set independently as run-time parameters.

We port the routines involving massive floating-point operations to GPUs such as the SRHD solvers and time-step calculations. On the other hand, we use CPUs to perform ghost-zone interpolation and patch refinement. As a result, we recommend using the refinement criteria only involving conserved variable for better performance because conserved variables are readily available from memory. By contrast, primitive variables can only be obtained by root-finding iteration, which is computationally expensive.

For enhancing software portability and reusability, GAMER not only supports both CPU-only and GPU modes but also allows the same physics modules to be shared by both CPU and GPU computations. Specifically, in the CPU-only mode, we compute different grid patches in the same MPI process in parallel with OpenMP. In the GPU mode, we replace these OpenMP parallel clauses with CUDA thread blocks and then use threads within the same thread block to update all cells within the same grid patch. This scheme maximizes the reuse of physics routines, avoids redundant code development and maintenance, and significantly lowers the barrier of code extension, especially for developers not acquainted with GPU programming. We have utilized this CPU/GPU integration infrastructure in the SRHD implementation.

\textsc{gamer-sr} supports the MUSCL-Hancock \citep{Toro} and VL (\citealt{VL1}; \citealt{VL2}) schemes for numerical integrations and both piece-wise linear method (PLM; \citealt{van_Leer_1979}) and piecewise parabolic method (PPM; \citealt{Woodward1984}) for data reconstruction. For the Riemann solver, it supports both relativistic HLLC and HLLE solvers (\citealt{HLLC_srhydro}; \citealt{HLLC_srmhd}), which have been adapted not only to be compatible with the TM EoS by using the corresponding sound speed, \Cref{sound_speed}, but also to evolve the reduced energy density (i.e. replacing $E$ with $\tilde{E}+Dc^2$).


\section{Flexible Time-step}
\textsc{gamer-sr} provides two Courant-Friedrichs-Lewy (CFL) conditions for time-step determination. The first one is based on the local signal propagation speed, $S_{\text{max}}$, which gives maximum allowed time-steps in a wide dynamical range. Thus, it can significantly improve performance when the maximum $v/c$ is not close to unity. The other is based on the speed of light, where we simply replace $S_{\text{max}}$ by $c$. It gives the most conservative estimation of time-steps and is more time-consuming when the flow speed is far less than $c$, although it is simple to implement and requires less computation.

To calculate $S_{\text{max}}$, we first define $\mathbf{\hat{u}_s}$ to be a spatial unit vector in the direction of sound propagation, we then apply the Lorentz boost with velocity $-\pmb{\beta}$ to the four-velocity of sound speed $(\gamma_s, U_s\mathbf{\hat{u}_s})$ from local rest frame to laboratory frame. We finally obtain the four-velocity of signal that travels in laboratory frame as follows:
\begin{equation}
    \left(\gamma\gamma_s+\gamma U_s\left(\pmb{\beta}\cdot\mathbf{\hat{u}_s}\right),U_{s}\mathbf{\mathbf{\hat{u}_s}}+\left(\gamma-1\right)U_s\left(\pmb{\hat{\beta}}\cdot\mathbf{\hat{u}_s}\right)\pmb{\hat{\beta}}+\pmb{\beta}\gamma\gamma_s\right),
    \label{time_step_propagation of information}
\end{equation}
where $\gamma$ and $\gamma_s$ are the Lorentz factor of flow and of sound speed. $U_s$ is the four-velocity of sound speed defined by $c_s/\sqrt{1-c_s^2}$. Since the direction of the fastest signal propagation is in general parallel to flow velocity, we assume that both sound and flow propagate in the same direction (i.e. $\mathbf{\hat{u}_s}=\pmb{\hat{\beta}}$). The spatial components of \Cref{time_step_propagation of information} then reduce to
\begin{equation}
    \left(\beta \gamma \gamma_s+\gamma U_s\right)\pmb{\hat{\beta}}.
    \label{time_step_x}
\end{equation}
Motivated by \Cref{time_step_x}, we simply choose $\abs{U^{i}}\gamma_s+\gamma U_s$ as the bound of each spatial component and sum over $\abs{U^{i}}\gamma_s+\gamma U_s$ for each spatial component to obtain
\begin{equation}
U_{\text{max}}
=\gamma_{s}\left(\abs{U_{x}}+\abs{U_{z}}+\abs{U_{z}}\right)+3\gamma U_s,
\label{time_step_U_max}
\end{equation}
where $U_{x/y/z}$ is the $x/y/z$-component of the four-velocities of flow.

Note that \Cref{time_step_U_max} is essentially the addition of flow speed and sound speed in special relativity theory. Converting \Cref{time_step_U_max} back to three-velocity

\begin{equation}
S_{\text{max}}=U_{\text{max}}/\sqrt{1+\left(U_{\text{max}}/c\right)^2}, \label{SMax}
\end{equation}

and substituting \Cref{SMax} into the CFL condition, we finally obtain the flexible time-step based on the local signal propagation speed for SRHD:
\begin{equation}
\label{time_step_sound_speed}
\Delta t = C_{\text{CFL}}\left(\frac{\Delta h}{S_{\text{max}}}\right),
\end{equation}
where $\Delta h$ is the cell spacing and $C_{\text{CFL}}$ the safety factor with a typical value of $\sim 0.5$ for MUSCL-Hancock and VL schemes.

Note that \Cref{time_step_sound_speed} can be reduced to its non-relativistic counterpart,
\begin{equation}
    \Delta t=C_{\text{CFL}}\left(\frac{\Delta h}{\abs{v_x}+\abs{v_y}+\abs{v_z}+3c_s}\right),
\end{equation}
when $\gamma\sim 1$ and to $\Delta t = C_{\text{CFL}}\Delta h/c$ when $\gamma \gg 1$.


\section{Handling unphysical results}
Unphysical results, for example, negative pressure, negative density and superluminal motion, can stem from the failure of the following criterion:
\begin{equation}
\left(\frac{\tilde{E}}{Dc^2}\right)^2+2\left(\frac{\tilde{E}}{Dc^2}\right)-\left(\frac{\abs{\mathbf{M}}}{Dc}\right)^2>\epsilon_{\text{machine}},
\label{Unphysical result}
\end{equation}
where the left-hand side involves the numerically updated quantities and $\epsilon_{\text{machine}}$ is the machine epsilon -- typically, $2\times10^{-16}$ for double precision and $1\times 10^{-7}$ for single precision. The failure may take place in one of the following four steps:

\emph{(1) SRHD solver}\\
SRHD solver is responsible for updating the conserved variables by a given time-step. If unphysical result occurs in a cell, we redo data reconstruction by reducing the original minmod coefficient by a factor of 0.75. If the failure still occurs, we further reduce the minmod coefficient repeatedly until \Cref{Unphysical result} passes or the reduced minmod coefficient vanishes. Note that interpolating with a vanished minmod coefficient is essentially equivalent to the piece-wise constant spatial reconstruction.

\emph{(2) Grid refinement}\\
Unphysical results may occur during grid refinement when performing interpolations on parent patches. The remedy here is the same as that in the SRHD solver. We repeat the interpolation process with a reduced minmod coefficient on the conserved variables until \Cref{Unphysical result} passes or the minmod coefficient vanishes. A vanished minmod coefficient is essentially equivalent to directly copying data from the parent patch without interpolation.

\emph{(3) Ghost-zone interpolation}\\
To preserve conservation, where the volume-weighted average of child patch data are equal to its parent patch data, we normally fill the ghost zones of the patches on level $\ell+1$ by interpolating the conserved variables on level $\ell$ when the ghost zones lie on level $\ell$. However, if unphysical results occur, we interpolate primitive variables instead. Interpolating primitive variables is more robust than interpolating conserved variables since \Cref{Unphysical result} is always satisfied. After interpolation, we fill the ghost zones with the conserved variables derived from the interpolated primitive variables. Note that this procedure still preserves conservation because ghost zones do not affect conservation.

\emph{(4) Flux correction operation}\\
For a leaf coarse patch adjacent to a coarse-fine interface, the flux difference between the coarse and fine patches on the interface will be used to correct the coarse-patch conserved variables adjacent to this interface. If unphysical results are found after this flux correction, we simply ignore the correction on the failed cells. Skipping the correction will break the strict conservation but it only occurs rarely.
