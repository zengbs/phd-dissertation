\chapter{Brief review of the \textit{Fermi}/eROSITA bubbles}
\label{c:bble-brief-review-of-bubbles}

%\section{Introduction}
The detection of the \textit{Fermi} bubbles \citep{Su2012,Ackermann2014,Narayanan2017},\
two large bubbles symmetrically extending about 50 degrees above and below the Galactic plane,\
is one of the great discoveries of the \textit{Fermi} Large Area Telescope \citep{Atwood2009}.\
The gamma-ray emission of the \textit{Fermi} bubbles is observed in the energy range\
of $\sim$1--100 GeV and has an almost spatially uniform hard\
spectrum, sharp edges and an approximately flat brightness distribution (see \citealt{Yang2018} for a review). Recently, the newly launched eROSITA \citep{Predehl2021} conducted an all-sky X-ray survey with high-spatial resolution and revealed two gigantic bubbles (eROSITA bubbles hereafter) extending to $\sim 80$ degrees in Galactic latitudes, corresponding to an intrinsic size of 14 kpc across \citep{Predehl2020}.\
The remarkable resemblance between the eROSITA and \textit{Fermi} bubbles suggest that they likely share the same origin \citep{Yang2022}.
Their symmetry about the GC further suggests that these Galactic bubbles may be generated by powerful energy injections from the GC, possibly related to nuclear star formation\
\citep{PhysRevLett.106.101102,Carretti2013,Crocker2015,Sarkar2015}
or past AGN activity \citep{Guo2012,Guo2012b,Yang2012,Yang2013,Mou2014,Yang2017}. The latter scenario is what we will focus on in this work.

Previous attempts \citep{Guo2012,Yang2012,Zhang2020}\
to model the formation of the symmetric Galactic bubbles by AGN jets have typically\
assumed that the jets are vertical to the Galactic plane. While there are some observational indications of pc-scale jets from Sgr A* that are found to be perpendicular to the Galactic plane \citep{Li, Zhu}, generally speaking, the AGN jet orientation is determined by the black hole spin and the accretion disk in the black-hole vicinity and does not need to align with the rotational axis of the host galaxy. Indeed, observationally there is a lack of evidence for the alignment between AGN jets and the disk normal \citep[e.g.,][]{Gallimore2006}. The jets are often oblique to the disk normal\
(e.g. NGC 3079, \citealt{Cecil2001}; NGC 1052, \citealt{Dopita2015}),\
and there are even cases in which the jets lie in the plane of the disk (e.g. IC 5063, \citealt{Morganti2015}).\

To this end, the aim of this work is to remove the assumption on jet orientations in the AGN jet models by introducing a dense, thin ISM disk\
that can interact with the central oblique jet,\
in an attempt to resolve the symmetry problem of the Galactic bubbles. More specifically, we use 3D\
SRHD simulations\
involving CR jet injections from the central supermassive black hole (SMBH) in the Galaxy to investigate\
whether the \textit{oblique} jet scenario is able to produce\
the \textit{symmetric} Galactic bubbles. We will verify whether the oblique jet model is\
consistent with the observed features of the Galactic bubbles, including the shape, \
surface brightness, and spectra of the \textit{Fermi} bubbles \citep{Ackermann2014}\
and microwave haze \citep{Dobler_2008,PlanckCollaborationIX2013}.

This paper is organized as follows.\
In Section \ref{Methodology}, we describe the numerical techniques and initial conditions employed.\
In Section \ref{Results}, we first present characteristics of our simulated Galactic bubbles,\
and then discuss how the disk affects the formation of the bubbles.\
We compare the morphology and profiles of the simulated eROSITA bubbles with\
the observed X-ray map in Section \ref{X-ray},\
and present the simulated and observed multi-wavelength spectra of\
the \textit{Fermi} bubbles and microwave haze in Section \ref{sec:gamma-ray-microwave}.\
Finally, the summary and implications of our findings are given in Section \ref{Conclusions}.

